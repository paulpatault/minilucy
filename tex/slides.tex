\documentclass{beamer}
\usepackage[abeamer, tikz, code]{ppbase}

\title{Projet Minilucy}
\date{\today}

\author{Paul Patault \& Émilien Lemaire}
\institute{ENS Paris-Saclay}

\lstset{
    style=mystyle,
    basicstyle=\small\ttfamily\color{craneblue!70},
    language=gospel,
    keywordstyle=\bfseries\color{craneblue!90},
}

\usepackage[sfdefault]{FiraSans}

\begin{document}
\maketitle


\begin{frame}{Features réalisés}
  \begin{itemize}[label=$\vcenter{\hbox{\scriptsize$\bullet$}}$]
    \item \texttt{when}
    \item \texttt{merge}
    \item \texttt{reset}
    \item \texttt{automates} (en surface uniquement)
  \end{itemize}
\end{frame}

\begin{frame}[fragile]{Construction \texttt{when}}
\end{frame}
\begin{frame}[fragile]{Schéma de compilation}
     % \scriptsize 
%      \begin{verbatim}
% Warning: this pattern-matching is not exhaustive.
% Here is a case that is not matched:
%    Node (Node _, _, _)
    %      \end{verbatim}
    \scalebox{0.75}{
\usetikzlibrary {arrows.meta}

\begin{tikzpicture}
  [auto,
   every path/.style ={draw, ultra thick,
       ->, % -Stealth[], 
       shorten >=1pt, line width=1pt},
   every node/.style ={thick, align=center, rounded corners, rectangle,
     minimum height=3.5em, minimum width=7em},
   state/.style  ={ draw=blue2,  fill=blue2!70},
   ]

  \node [state]                (ulus) {lustre} edge [loop below] ();
  \node [state, right=of ulus] (tlus) {typed lustre};
  \node [state, right=of tlus] (nlus) {normalised lustre};
  \node [state, right=of nlus] (clus) {clocked lustre};
  \node [state, below=of clus] (slus) {scheduled lustre};
  \node [state, below=of slus] (impl) {IMP};
  \node [state, left =of impl] (c) {C};

  \begin{scope}
    \path (ulus) -- (tlus);
    \path (tlus) -- (nlus);
    \path (nlus) -- (clus);
    \path (clus.south) -- (slus.north);
    \path (slus.south) -- (impl.north);
    \path (impl.west)  -- (c.east);
  \end{scope}
\end{tikzpicture}
    }
\end{frame}



% \begin{frame}{All small caps}
%     This frame uses the \texttt{allsmallcaps} titleformat.

%     \begin{block}{Potential problems}
%         As a rule of thumb: Just use it for plaintext-only titles.
%     \end{block}
%     \begin{alertblock}{Potential problems}
%         As a rule of thumb: Just use it for plaintext-only titles.
%     \end{alertblock}
%     \begin{exampleblock}{Potential problems}
%     \end{exampleblock}
% \end{frame}

\end{document}
