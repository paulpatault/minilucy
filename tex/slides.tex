\documentclass{beamer}
\usepackage[abeamer, tikz, code]{ppbase}

\title{Projet Minilucy}
\date{\today}

\author{Paul Patault \& Émilien Lemaire}
\institute{ENS Paris-Saclay}

\lstset{
    style=mystyle,
    basicstyle=\small\ttfamily\color{craneblue!70},
    language=gospel,
    keywordstyle=\bfseries\color{craneblue!90},
}

\usepackage[sfdefault]{FiraSans}

\begin{document}
\maketitle


\begin{frame}{Features réalisés}
  \begin{itemize}[label=$\vcenter{\hbox{\scriptsize$\bullet$}}$]
    \item \texttt{when}
    \item \texttt{merge}
    \item \texttt{reset}
    \item \texttt{automates} (en surface uniquement)
  \end{itemize}
\end{frame}

\begin{frame}[fragile]{Construction \texttt{when}}
     \scriptsize
     \begin{verbatim}
Warning: this pattern-matching is not exhaustive.
Here is a case that is not matched:
   Node (Node _, _, _)
     \end{verbatim}
\end{frame}

% \begin{frame}{All small caps}
%     This frame uses the \texttt{allsmallcaps} titleformat.

%     \begin{block}{Potential problems}
%         As a rule of thumb: Just use it for plaintext-only titles.
%     \end{block}
%     \begin{alertblock}{Potential problems}
%         As a rule of thumb: Just use it for plaintext-only titles.
%     \end{alertblock}
%     \begin{exampleblock}{Potential problems}
%     \end{exampleblock}
% \end{frame}

\end{document}
