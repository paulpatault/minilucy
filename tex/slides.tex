\documentclass{beamer}
\usepackage[abeamer, tikz, code]{ppbase}

\title{Projet Minilucy}
\date{\today}

\author{Paul Patault \& Émilien Lemaire}
\institute{ENS Paris-Saclay}

\lstset{
    basicstyle=\scriptsize\ttfamily\color{craneblue!70},
    language=lustre,
    keywordstyle=\bfseries\color{craneblue!90},
}

\usepackage[sfdefault]{FiraSans}

\newcommand{\noeud}{n\oe{}ud\ }
\renewcommand{\tt}[1]{\texttt{\color{craneblue!80}#1}}

\AtBeginSection[ ]
{
\begin{frame}{}
    \tableofcontents[currentsection]
    \thispagestyle{empty}
\end{frame}
}

%--------------------------------------------------------------------------------------------------

\begin{document}
\maketitle

%--------------------------------------------------------------------------------------------------

\section{idée générale}

\begin{frame}[fragile]{Schéma de compilation}
    \scalebox{0.7}{

        \usetikzlibrary {arrows.meta}

\begin{tikzpicture}
    [auto,
        every path/.style ={draw, ultra thick, ->, % -Stealth[],
                            shorten >=1pt, line width=1pt},
        every node/.style ={thick, align=center, rounded corners, rectangle, minimum height=3.5em, minimum width=7em},
        state/.style  ={ draw=blue2,  fill=blue2!70},
        stateX/.style ={ draw=orange2,  fill=orange2!70},
    ]

  \node [stateX]                (plus) {parsed lustre};
  \node [state, below=of plus] (ulus) {un-sugared lustre};
  \node [state, right=of ulus] (tlus) {typed lustre};
  \node [state, right=of tlus] (nlus) {normalised lustre};
  \node [state, right=of nlus] (clus) {clocked lustre};
  \node [state, below=of clus] (slus) {scheduled lustre};
  \node [state, below=of slus] (impl) {IMP};
  \node [stateX, left =of impl] (c) {C};

  \begin{scope}
    \path (plus.south) -- (ulus.north);
    \path (ulus) -- (tlus);
    \path (tlus) -- (nlus);
    \path (nlus) -- (clus);
    \path (clus.south) -- (slus.north);
    \path (slus.south) -- (impl.north);
    \path (impl.west)  -- (c.east);
  \end{scope}
\end{tikzpicture}
    }
\end{frame}

\lstset{
    basicstyle=\scriptsize\ttfamily\color{craneblue!70},
}

%--------------------------------------------------------------------------------------------------

\begin{frame}{Features réalisées}
    \begin{itemize}[label=$\vcenter{\hbox{\scriptsize$\bullet$}}$]
        \item \texttt{when}
        \item \texttt{merge}
        \item \texttt{reset}
        \item \texttt{automates} (en surface uniquement)
    \end{itemize}
\end{frame}

%--------------------------------------------------------------------------------------------------

\section{when}

\begin{frame}[fragile]{Vérification des horloges}

\begin{lstlisting}
node clock_error (c: bool) returns (o: int);
  var x: int;
let
  x = 0 when True(c);
  o = x;
tel
\end{lstlisting}

 {
     \scriptsize \color{red!80}
     \begin{verbatim}
      Clocking error: The expected clock is Base, got Base on True(c)
     \end{verbatim}
}


\end{frame}

%--------------------------------------------------------------------------------------------------

\section{automate}

\begin{frame}[fragile]{Traduction de l'automate}

\begin{lstlisting}
node syracuse (i: int) returns (o: int);
let
  automaton
  | Even ->
      o = i -> pre o / 2;
      until (o mod 2 = 1) continue Odd
  | Odd ->
      o = i -> pre o * 3 + 1;
      until (o mod 2 = 0) continue Even
  end
tel
\end{lstlisting}

\end{frame}

\begin{frame}[fragile]{Traduction de l'automate}

\begin{lstlisting}
type t = Even | Odd

node syracuse (i: int) returns (o: int);
    var state: t; cond__4: bool; cond__3: bool;
let
  cond__4 = o mod 2 = 0;
  cond__3 = o mod 2 = 1;
  state = Even ->
           pre (merge state
                 (Even -> if cond__3 then Odd else Even)
                 (Odd -> if cond__4 then Even else Odd));
  o = merge state
        (Even -> i -> pre o / 2)
        (Odd  -> i -> pre o * 3 + 1);
tel
\end{lstlisting}

\end{frame}

\lstset{ basicstyle=\tiny\ttfamily\color{craneblue!70} }

\begin{frame}[fragile]{Automate (une slide)}
\begin{tikzpicture}[remember picture, overlay]

    \draw[DarkerGrey, line width=0.7mm, loosely dotted] (4,3.5) -- (4,-3.5);

    \node[above left, xshift=+13em, yshift=+7em] at (current page.south west)
        {\begin{minipage}{.4\textwidth}
\begin{lstlisting}
node syracuse (i: int)
  returns (o: int);
let
  automaton
  | Even ->
      o = i -> pre o / 2;
      until (o mod 2 = 1)
      continue Odd
  | Odd ->
      o = i -> pre o * 3 + 1;
      until (o mod 2 = 0)
      continue Even
  end
tel
\end{lstlisting}
        \end{minipage}};
    \node[above left, xshift=+1em, yshift=+5.5em] at (current page.south east)
        {\begin{minipage}{.7\textwidth}
\begin{lstlisting}
type t = Even | Odd

node syracuse(i: int)
  returns (o: int);
  var state: t;
      cond__4, cond__3: bool;
let
  cond__4 = o mod 2 = 0;
  cond__3 = o mod 2 = 1;
  state = Even ->
           pre (merge state
                 (Even ->
                   if cond__3 then Odd else Even)
                 (Odd ->
                   if cond__4 then Even else Odd));
  o = merge state
        (Even -> i -> pre o / 2)
        (Odd  -> i -> pre o * 3 + 1);
tel
\end{lstlisting}
        \end{minipage}};
 \end{tikzpicture}

\end{frame}

%--------------------------------------------------------------------------------------------------

\section{merge}

\lstset{ basicstyle=\scriptsize\ttfamily\color{craneblue!70}, }

\begin{frame}[fragile]{Compilation des « merge »}

\begin{lstlisting}
type t = A | B

node main0 () returns (o: t);
let
  o = A't fby
        (merge o
          (A't -> B't when A(o))
          (B't -> A't when B(o)));
tel
\end{lstlisting}

\end{frame}

\begin{frame}[fragile]{Compilation des « merge »}

\begin{lstlisting}[language=c]
enum t { A, B };
// ...
enum t main0(...) {
    // ...
    switch (o) {
        case A: {
            res = B;
            break;
        }
        case B: {
            res = A;
            break;
        }
    };
    // ...
    return res;
}
\end{lstlisting}

\end{frame}

\lstset{ basicstyle=\tiny\ttfamily\color{craneblue!70} }

\begin{frame}[fragile]{Compilation des « merge »}
\begin{tikzpicture}[remember picture, overlay]

    \draw[DarkerGrey, line width=0.7mm, loosely dotted] (5.4,3.5) -- (5.4,-3.5);

    \node[above left, xshift=+15em, yshift=+7em] at (current page.south west)
        {\begin{minipage}{.45\textwidth}
\begin{lstlisting}
type t = A | B

node main0 () returns (o: t);
let
  o = A't fby
        (merge o
          (A't -> B't when A(o))
          (B't -> A't when B(o)));
tel
\end{lstlisting}
        \end{minipage}};
    \node[above left, xshift=+6em, yshift=+5em] at (current page.south east)
        {\begin{minipage}{.7\textwidth}
\begin{lstlisting}[language=c]
enum t { A, B };
// ...
enum t main0(...) {
    // ...
    switch (o) {
        case A: {
            res = B;
            break;
        }
        case B: {
            res = A;
            break;
        }
    };
    // ...
    return res;
}
\end{lstlisting}
        \end{minipage}};
 \end{tikzpicture}

\end{frame}


%--------------------------------------------------------------------------------------------------

\section{mémoire}

\begin{frame}[fragile]{Gestion de la mémoire}
    on créé une mémoire pour chaque \noeud si :
    \begin{itemize}[label=$\vcenter{\hbox{\scriptsize$\bullet$}}$]
        \item présence d'un \texttt{fby} dans le \noeud
        \item appel d'un \noeud qui a une mémoire
        \item présence d'un automate
    \end{itemize}

\end{frame}

\lstset{ basicstyle=\scriptsize\ttfamily\color{craneblue!70} }

\begin{frame}[fragile]{Gestion de la mémoire}

\begin{tikzpicture}[remember picture, overlay]

    \draw[DarkerGrey, line width=0.7mm, loosely dotted] (5.4,3.5) -- (5.4,-3.5);

    \node[right, xshift=+1em] at (current page.west)
        {\begin{minipage}{.5\textwidth}
\begin{lstlisting}
node f () returns (o:int);
let
  o = 1;
tel

node main0 () returns (o:int);
let
  o = f();
tel
\end{lstlisting}
        \end{minipage}};
    \node[left, xshift=-1em] at (current page.east)
        {\begin{minipage}{.5\textwidth}
\begin{lstlisting}[escapeinside={\$}{\$}]
node f () returns (o:int);
let
  o = 1 $\color{red}fby 2$;
tel

node main0 () returns (o:int);
let
  o = f();
tel
\end{lstlisting}
        \end{minipage}};
 \end{tikzpicture}

\end{frame}

\begin{frame}[fragile]{Gestion de la mémoire}

\begin{lstlisting}[language=c]
struct f_mem {
  int o;
};
struct main0_mem {
  struct f_mem f_next;
};
//...
void f_init (struct f_mem* mem) {
  mem->o = 1;
}
//...
void main0_init (struct main0_mem* mem) {
  f_init(&(mem->f_next));
}
//...
\end{lstlisting}

\end{frame}

\lstset{ basicstyle=\tiny\ttfamily\color{craneblue!70} }

\begin{frame}[fragile]{Gestion de la mémoire}
\begin{tikzpicture}[remember picture, overlay]

    \draw[DarkerGrey, line width=0.7mm, loosely dotted] (4.4,3.5) -- (4.4,-3.5);

    \node[above left, xshift=+13em, yshift=+7em] at (current page.south west)
        {\begin{minipage}{.4\textwidth}
\begin{lstlisting}
node f () returns (o:int);
let
  o = 1 fby 2;
tel

node main0 () returns (o:int);
let
  o = f();
tel
\end{lstlisting}
        \end{minipage}};
    \node[above left, xshift=+3em, yshift=+5.5em] at (current page.south east)
        {\begin{minipage}{.7\textwidth}
\begin{lstlisting}[language=c]
struct f_mem {
  int o;
};
struct main0_mem {
  struct f_mem f_next;
};
//...
void f_init (struct f_mem* mem) {
  mem->o = 1;
}
//...
void main0_init (struct main0_mem* mem) {
  f_init(&(mem->f_next));
}
//...
\end{lstlisting}
        \end{minipage}};
 \end{tikzpicture}

\end{frame}


%--------------------------------------------------------------------------------------------------
\section{reset}

\lstset{ basicstyle=\scriptsize\ttfamily\color{craneblue!70}, }

\begin{frame}[fragile]{Compilation des « reset »}

\begin{lstlisting}
node incr () returns (cpt: int);
let
  cpt = 0 fby cpt + 1;
tel

node main0 (i: bool) returns (o: int);
let
  o = reset incr() every i;
tel
\end{lstlisting}

\medskip

$\to$ ajouter un appel à \tt{incr\_init()} quand \tt{i} est vrai

\end{frame}

\begin{frame}
   \centerline{Démo !}
   \thispagestyle{empty}
\end{frame}

% \begin{frame}{All small caps}
%     This frame uses the \texttt{allsmallcaps} titleformat.

%     \begin{block}{Potential problems}
%         As a rule of thumb: Just use it for plaintext-only titles.
%     \end{block}
%     \begin{alertblock}{Potential problems}
%         As a rule of thumb: Just use it for plaintext-only titles.
%     \end{alertblock}
%     \begin{exampleblock}{Potential problems}
%     \end{exampleblock}
% \end{frame}

\end{document}
