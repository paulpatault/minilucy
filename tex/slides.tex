\documentclass{beamer}
\usepackage[abeamer, tikz, code]{ppbase}

\title{Projet Minilucy}
\date{\today}

\author{Paul Patault \& Émilien Lemaire}
\institute{ENS Paris-Saclay}

\lstset{
    basicstyle=\scriptsize\ttfamily\color{craneblue!70},
    language=lustre,
    keywordstyle=\bfseries\color{craneblue!90},
}

\usepackage[sfdefault]{FiraSans}

\begin{document}
\maketitle


\begin{frame}{Features réalisés}
    \begin{itemize}[label=$\vcenter{\hbox{\scriptsize$\bullet$}}$]
        \item \texttt{when}
        \item \texttt{merge}
        \item \texttt{reset}
        \item \texttt{automates} (en surface uniquement)
    \end{itemize}
\end{frame}

\begin{frame}[fragile]{Traduction de l'automate}

\begin{lstlisting}
node syracuse (i: int) returns (o: int);
let
  automaton
  | Even ->
      o = i -> pre o / 2;
      until (o mod 2 = 1) continue Odd
  | Odd ->
      o = i -> pre o * 3 + 1;
      until (o mod 2 = 0) continue Even
  end
tel
\end{lstlisting}

\end{frame}

\begin{frame}[fragile]{Traduction de l'automate}

\begin{lstlisting}
type t = Even | Odd

node syracuse(i : int) returns (o : int);
    var state : t; cond__4 : bool; cond__3 : bool;
let
  cond__4 = o mod 2 = 0;
  cond__3 = o mod 2 = 1;
  state = Even ->
           pre (merge state
                 (Even -> if cond__3 then Odd else Even)
                 (Odd -> if cond__4 then Even else Odd));
  o = merge state
        (Even -> i -> pre o / 2)
        (Odd  -> i -> pre o * 3 + 1);
tel
\end{lstlisting}

\end{frame}

\lstset{
    basicstyle=\tiny\ttfamily\color{craneblue!70},
    language=lustre,
    keywordstyle=\bfseries\color{craneblue!90},
}

\begin{frame}[fragile]{Automate (une slide)}
\begin{tikzpicture}[remember picture, overlay]

    \node[above left, xshift=+13em, yshift=+7em] at (current page.south west)
        {\begin{minipage}{.4\textwidth}
\begin{lstlisting}
node syracuse (i: int)
  returns (o: int);
let
  automaton
  | Even ->
      o = i -> pre o / 2;
      until (o mod 2 = 1)
      continue Odd
  | Odd ->
      o = i -> pre o * 3 + 1;
      until (o mod 2 = 0)
      continue Even
  end
tel
\end{lstlisting}
        \end{minipage}};
    \node[above left, xshift=+1em, yshift=+5.5em] at (current page.south east)
        {\begin{minipage}{.7\textwidth}
\begin{lstlisting}
type t = Even | Odd

node syracuse(i : int)
  returns (o : int);
  var state : t;
      cond__4, cond__3 : bool;
let
  cond__4 = o mod 2 = 0;
  cond__3 = o mod 2 = 1;
  state = Even ->
           pre (merge state
                 (Even ->
                   if cond__3 then Odd else Even)
                 (Odd ->
                   if cond__4 then Even else Odd));
  o = merge state
        (Even -> i -> pre o / 2)
        (Odd  -> i -> pre o * 3 + 1);
tel
\end{lstlisting}
        \end{minipage}};
 \end{tikzpicture}

\end{frame}

\begin{frame}[fragile]{Schéma de compilation}
    % \scriptsize 
    %      \begin{verbatim}
% Warning: this pattern-matching is not exhaustive.
% Here is a case that is not matched:
%    Node (Node _, _, _)
    %      \end{verbatim}
    \scalebox{0.75}{

        \usetikzlibrary {arrows.meta}

\begin{tikzpicture}
    [auto,
        every path/.style ={draw, ultra thick, ->, % -Stealth[],
                            shorten >=1pt, line width=1pt},
        every node/.style ={thick, align=center, rounded corners, rectangle, minimum height=3.5em, minimum width=7em},
        state/.style ={ draw=blue2,  fill=blue2!70},
    ]

  \node [state]                (ulus) {lustre} edge [loop below] ();
  \node [state, right=of ulus] (tlus) {typed lustre};
  \node [state, right=of tlus] (nlus) {normalised lustre};
  \node [state, right=of nlus] (clus) {clocked lustre};
  \node [state, below=of clus] (slus) {scheduled lustre};
  \node [state, below=of slus] (impl) {IMP};
  \node [state, left =of impl] (c) {C};

  \begin{scope}
    \path (ulus) -- (tlus);
    \path (tlus) -- (nlus);
    \path (nlus) -- (clus);
    \path (clus.south) -- (slus.north);
    \path (slus.south) -- (impl.north);
    \path (impl.west)  -- (c.east);
  \end{scope}
\end{tikzpicture}
    }
\end{frame}



% \begin{frame}{All small caps}
%     This frame uses the \texttt{allsmallcaps} titleformat.

%     \begin{block}{Potential problems}
%         As a rule of thumb: Just use it for plaintext-only titles.
%     \end{block}
%     \begin{alertblock}{Potential problems}
%         As a rule of thumb: Just use it for plaintext-only titles.
%     \end{alertblock}
%     \begin{exampleblock}{Potential problems}
%     \end{exampleblock}
% \end{frame}

\end{document}
